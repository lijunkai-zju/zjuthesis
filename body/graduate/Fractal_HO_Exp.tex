\chapter{分形BBH模型的有限元仿真和实验观测}
直观上,分形晶格会关闭拓扑保护\cite{liu2021sierpinski},因为缺乏明确的体和随后的体-边对应关系。然而,最近的一项报告显示,Floquet拓扑分形绝缘体可以在由螺旋光波导组成的Sierpinski垫片的光子晶格中实现\cite{yang2020photonic}。实验观察\cite{biesenthal2022fractal}展示了分形中一阶拓扑绝缘体的第一个例子。然而,迄今为止,对于分形高阶拓扑绝缘体的实验实现尚未完成。

在实验中,我们利用声学谐振器 [48,49] 设计了上述晶格模型,并测量了系统的局部响应。通过遍历所有晶格点,我们实验观测到了外角和内角处丰富的拓扑角态,其维度分别对应于 0 和 1.89。角态的分数维度此前从未被报道,因此使得我们的研究与传统拓扑绝缘体明显不同。

高阶拓扑绝缘体相比于一阶拓扑绝缘体支持更低维度的拓扑边界态,并已在整数维系统中被广泛研究。在此,我们通过实验在声学分形晶格中展示了一种新的高阶拓扑相,从而提供了一种新的范式。通过在方形晶格中引入分形性,我们揭示了一个受限的高阶相位图,其中包含丰富的角态,包括零维外角态和 1.89 维内角态。因此,该系统的余维数为 1.89,该分形模型可归类为分数阶拓扑绝缘体。计算出的外/内角处的分数量子化电荷表明,在分形系统中的所有角态均为拓扑非平庸态。最后,在一个制备的分形晶格中,我们通过局部声学测量实验观察到了外/内角态。我们的研究展示了声学分形晶格中的高阶拓扑相,并可能为分数阶拓扑绝缘体的发展铺平道路。

\section{分形BBH模型的有限元仿真}
\subsection{声学负耦合}
\subsection{声学BBH模型原胞}
\subsection{声学分形晶格频谱}

\section{声学分形BBH模型的观测}
\subsection{实验装置}
\subsection{声学局域态密度}
\subsection{拓扑态的实验观测}
\subsection{实验频谱与仿真频谱的对应}
