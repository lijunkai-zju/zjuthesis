\chapter{总结与展望}
拓扑绝缘体这一概念自1980年整数量子霍尔效应发现后,在过去40年间诞生了丰富的理论和实验突破。本文围绕分形拓扑绝缘体的理论建模与实验实现展开系统性研究,结合凝聚态物理中的拓扑相理论、分形几何学和非平衡系统动力学,揭示了分形维度与拓扑保护边界态之间的深刻关联。本文第一章回顾了量子霍尔效应,反常量子霍尔效应,分数量子霍尔效应以及高阶拓扑绝缘体相应的基本理论以及其物理性质。在第二章中,笔者介绍了自相似性,分形维度,随机分形等分形拓扑绝缘体的基本理论,并介绍了分形拓扑绝缘体的理论模型和相应的实验进展。在第三章中,笔者实现了谢宾斯基三角形上的霍尔丹模型,其具有Cantor函数形式的分形能谱,并展示了一个压缩的拓扑相图。在第四章,笔者实现了谢宾斯基地毯上的BBH模型,其具有1.89维的晶格,零维的外角态与1.89维的内角态,这些态共享同样的压缩拓扑相区域。在第五章中,笔者通过有限元仿真和直接实验观测,制备和观测了分形霍尔丹模型和分形BBH模型的拓扑态。

尽管分形拓扑绝缘体的研究已取得初步进展,其理论体系的完善和实验验证仍面临诸多挑战。未来研究方向可围绕以下几个方面展开:

第一个研究方向是分形拓扑绝缘体普适理论。当前研究受限于分形晶格平移对称性缺失导致的波矢空间失效问题,传统基于动量空间的拓扑分类方案面临根本性挑战。需要发展基于非交换几何的实空间拓扑不变量理论,将Bott指数、实空间陈数等工具推广至Hausdorff维度空间。特别地,分形晶格的无限层级嵌套特性要求建立超越有限尺度近似的普适理论框架,这可能需要引入分数阶微分几何与分形流形的新数学工具。

第二个研究方向是精确的维度阈值。在本文,笔者展示了分数维度对拓扑相图的普遍压缩性质。这种压缩性质表明定义在二维平面的拓扑相被放置在分数维度的晶格上时,其拓扑鲁棒性会减弱。此外,在不断降低量子霍尔效应的维度时,出现了拓扑相被关断的情况。这暗示拓扑会随着维度降低而减弱甚至消失,并存在一个相应的维度阈值。然而一个精确的维度阈值仍然缺乏。

第三个研究方向是分形-拓扑-强关联的交叉效应。在强关联电子体系中,分形晶格的几何阻挫与拓扑序的相互作用可能催生新型量子自旋液体或分形超导态。该方向的研究需要发展适用于分数维度的数值方法,如基于张量网络的分形密度矩阵重整化群算法,以及考虑分形度规的量子蒙特卡罗模拟技术。

第四个研究方向是随机分形的拓扑效应。随机分形更广阔的出现于自然界和日常生活中,拓扑相在随机分形上的特征可能会给拓扑带来更广阔的应用。分形拓扑绝缘体的多尺度边缘态为光子芯片集成、声学超表面设计、高密度信息存储和分形天线设计提供了新思路,其应用潜力亟待深入挖掘。此外在随机分形上的拓扑态研究还可能对拓扑准晶的研究提供启发。

分形拓扑绝缘体研究架起了凝聚态物理与分形几何的桥梁,其核心科学价值在于揭示维度分数化对拓扑量子现象的影响。本文通过理论计算和实验观测验证了分形拓扑绝缘体并讨论了例如压缩相图,与晶格一致的拓扑态维度等一系列新奇性质。分数维度作为一个全新自由度将对拓扑物理带来全新的进展。