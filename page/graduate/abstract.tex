\cleardoublepage
\chapternonum{摘要}
分形具有迷人的特性,不仅在于其优美几何的吸引力,还因为其所展现出的新颖性质。分形最显著的特点是具有非整数维度。这种分形维度提供了一种新的自由度,从而产生了许多新颖的特征,例如自相似性和尺度不变性。分形特性已在许多领域被广泛研究。另一方面,以上世纪80年代发现的量子霍尔效应为代表的拓扑相变激发出了大量研究兴趣。然而拓扑物理植根于整数维度。依赖对称性的拓扑十重分类强烈依赖于维度,但仅包含了整数维度的结果。最近,在分数维度中实现拓扑态方面取得了一系列引人注目的进展。但拓扑绝缘体是否能在分数维度实现仍然存在争议:分形光子拓扑绝缘体被实验观测到,但也有在电子系统中分形晶格造成的拓扑现象关断的证据。不仅如此,分数维度对拓扑物理的影响也处于早期阶段。本论文共分为六个章节。第一章阐述了拓扑物理学的经典模型及其相应的物理特征。第二章回顾了分形几何的基本理论,并对分形拓扑绝缘体的理论与实验研究进展进行了综述。第三章在谢宾斯基三角形结构中实现了霍尔丹模型,揭示了该体系具有Cantor函数形式的分形能谱以及压缩的拓扑相图。第四章在谢宾斯基地毯结构中实现了Benalcazar-Bernevig-Hughes模型,展示了该体系所特有的分形蝴蝶能谱、与晶格维度一致的拓扑态,以及分裂的分数电荷。第五章通过数值仿真与实验研究验证了声学分形拓扑绝缘体的拓扑特性,并对其拓扑态进行了实验测量。第六章对全文工作进行了总结,展望了分形拓扑绝缘体未来可能的研究方向。

本论文的创新性工作成果如下:
\begin{itemize}
    \item \textbf{理论发现并实验证明了分数维度对于拓扑相图的压缩效应。}在分形几何上实现拓扑绝缘体取得了一系列突破性进展,越来越多的证据证明在分数维度实现拓扑物态是可能的。然而分形几何对于拓扑物理的影响仍然处于早期阶段。本文发现随着维度的降低,原本定义在二维空间的拓扑相出现了拓扑保护范围的缩小。基于这一思想,笔者进一步研究了谢宾斯基家族的各种分形结构,给出了拓扑现象一个模糊的维度阈值。
    \item \textbf{余维度为零的拓扑物态。}高阶拓扑相描述了那些晶格与拓扑态余维度大于一的拓扑相。在第四章的分形Benalcazar-Bernevig-Hughes模型中,分形晶格的空隙提供了丰富的内角态,这些内角态随着晶格迭代不断增加。通过盒计数计算,本文发现其内角态与谢宾斯基地毯晶格具有同样的维度1.89,这意味着其余维度为零。这一结果可能拓展传统的“高阶”拓扑绝缘体的含义。
    \item \textbf{分形晶格的分形能谱。}分形能谱最早在量子霍尔效应中被发现,也即著名的霍夫斯塔特蝴蝶。在第三章的分形霍尔丹模型中,出现了独特的分形能谱,其形式是广为人知的Cantor函数,也被称作“恶魔阶梯”,维度为0.3981。此外,在第四章的分形Benalcazar-Bernevig-Hughes模型中,也发现了分形蝴蝶能谱。
    \item \textbf{分形层级拓扑态。}在分形晶格中,会由于内部空缺产生不同层次的拓扑态。在分形Benalcazar-Bernevig-Hughes模型中,这些层级拓扑态会产生独特的分数电荷,其内角态可以产生大小为1.25的分数电荷。这种层级拓扑态的另一个新奇现象是在G(1)晶格中,由于不存在体格点因此不存在体态。其所有特征模式均为边缘态或角态。
\end{itemize}



\cleardoublepage
\chapternonum{Abstract}
Fractals exhibit fascinating characteristics, not only due to their captivating geometric beauty but also because of their novel properties. The most notable feature of fractals is their non-integer dimension, introducing a new degree of freedom that leads to numerous novel features, such as self-similarity and scale invariance. Fractal properties have been extensively studied across various fields. On the other hand, topological phase transitions, exemplified by the quantum Hall effect discovered in the 1980s, have sparked extensive research interest. However, topological physics traditionally relies on integer dimensions. The symmetry-based tenfold topological classification is strongly dimension-dependent but covers only integer-dimensional results. Recently, significant progress has been made in realizing topological states in fractional dimensions. Nevertheless, it remains controversial whether topological insulators can be realized in fractional dimensions: fractal photonic topological insulators have been experimentally observed, while evidence also suggests the breakdown of topological phenomena in fractal lattices in electronic systems. Furthermore, the impact of fractional dimensions on topological physics is still in its early stages.

This dissertation consists of six chapters. Chapter 1 introduces classical models of topological physics and their corresponding physical properties. Chapter 2 presents fundamental theories of fractal geometry and reviews theoretical and experimental developments in fractal topological insulators. Chapter 3 implements the Haldane model on a Sierpinski triangle, exhibiting a Cantor-function fractal energy spectrum and compressed topological phase diagram. Chapter 4 implements the Benalcazar-Bernevig-Hughes model on a Sierpinski carpet, demonstrating a fractal butterfly energy spectrum, topological states consistent with lattice dimension, and splitted fractional charges. Chapter 5 validates the topological properties of acoustic fractal topological insulators through simulations and experiments, measuring their topological states. Chapter 6 summarizes the dissertation and proposes future research directions for fractal topological insulators.

The innovative achievements of this dissertation are as follows:
\begin{itemize}
    \item \textbf{Theoretical discovery and experimental verification of fractional dimension-induced squeeze in topological phase diagrams.} Implementing topological insulators on fractal geometries has led to groundbreaking advances, with increasing evidence supporting the realization of topological states in fractional dimensions. However, the influence of fractal geometry on topological physics remains nascent. This work finds that, as dimensionality decreases, the topological phases originally defined in two-dimensional space exhibit reduced regions of topological protection. Based on this concept, various fractal structures from the Sierpinski family are further investigated, revealing an ambiguous dimensional threshold for topological phenomena.
    \item \textbf{Topological states with zero codimension.} Higher-order topological phases typically describe topological states with codimensions greater than one relative to the lattice. In the fractal Benalcazar-Bernevig-Hughes model (Chapter 4), gaps in fractal lattices yield abundant inner-corner states, increasing with lattice iterations. Through box-counting calculations, this dissertation identifies the inner-corner states as having the same dimension (1.89) as the Sierpinski carpet lattice, implying a codimension of zero. This result may expand the traditional meaning of "higher-order" topological insulators.
    \item \textbf{Fractal energy spectra in fractal lattices.} Fractal energy spectra were first discovered in the quantum Hall effect, known as the Hofstadter butterfly. Chapter 3's fractal Haldane model presents a distinctive fractal energy spectrum, taking the form of the well-known Cantor function, also called the "Devil's Staircase," with a dimension of 0.3981. Moreover, the fractal Benalcazar-Bernevig-Hughes model in Chapter 4 also exhibits a fractal butterfly energy spectrum.
    \item \textbf{Hierarchical topological states in fractal lattices.} Fractal lattices exhibit different hierarchical topological states due to internal vacancies. In the fractal Benalcazar-Bernevig-Hughes model, these hierarchical states produce unique fractional charges, with inner-corner states generating fractional charges of magnitude 1.25. Another novel phenomenon of these hierarchical states occurs in the G(1) lattice, where bulk states are absent due to the lack of bulk lattice sites. Consequently, all characteristic modes are either edge or corner states.
\end{itemize}

